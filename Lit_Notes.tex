\documentclass[landscape]{slides}
\usepackage[landscape]{geometry}
\usepackage{color}
\usepackage{amsfonts}
\usepackage{bm}

\begin{document}
	\begin{slide}
	\textcolor{red}{\Large{Colour Performance of Ceramic Nano-pigments}}\footnote{\cite{Cavalcante2009}}
	\begin{itemize}
		
		\small{\item CoAl$_2$4, \textcolor{cyan}{Cyan}
		\\Au, \textcolor{magenta}{Yellow} \\(Ti,Cr,Sb)O$_2$, \textcolor{yellow}{Yellow}
\\CoFe$_2$O$_4$, Black}
	
	\item Black CoFe$_2$O$_4$ a spinel (inverse) structure with Co$^{2+}$ in \textbf{both} tetrahedral and octahedral sites 
	
	\item ``Extensive Co-O and Fe-O charge
	transfers, together with d-d electron transitions of Co$^{2+}$ and Fe$^{3+}$ in multiple coordination,
	ensure the full absorption of the visible spectrum''
	
	\item Structure stable from 800--1200$^\circ$C (albeit with a growth in crystallite size from 20 to ~250 nm)
	
	\item Fast growth in size in glassy coatings but prevented in glazes (persists in 20-40 nm range)
	
	\item Black colour gets darker (without regular trend) with temperature (expressed in CIELAB color space by L* parameter)
\end{itemize}
	\end{slide}

\begin{slide}
	\textcolor{red}{\Large{Transition Metal Oxides; Rao \& Raveau}}
	\begin{itemize}
		\item 
	\end{itemize}
\end{slide}
\end{document}